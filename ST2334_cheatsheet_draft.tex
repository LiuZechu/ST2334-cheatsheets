%%%%%%%%%%%%%%%%%%%%%%%%%%%%%%%%%%%%%%%%%%%%%%%%%%%%%%%%%%%%%%%%%%%%%%
% Source: Dave Richeson (divisbyzero.com), Dickinson College
% 
% A one-size-fits-all LaTeX cheat sheet. Kept to two pages, so it 
% can be printed (double-sided) on one piece of paper
% 
% Feel free to distribute this example, but please keep the referral
% to divisbyzero.com
% 
% If you're new to LaTeX, the wikibook is a great place to start:
% http://en.wikibooks.org/wiki/LaTeX
%%%%%%%%%%%%%%%%%%%%%%%%%%%%%%%%%%%%%%%%%%%%%%%%%%%%%%%%%%%%%%%%%%%%%%

\documentclass[10pt,landscape]{article}
\usepackage{amssymb,amsmath,amsthm,amsfonts}
\usepackage{multicol,multirow}
\usepackage{calc}
\usepackage{ifthen}
\usepackage[landscape]{geometry}
\usepackage[colorlinks=true,citecolor=blue,linkcolor=blue]{hyperref}


\ifthenelse{\lengthtest { \paperwidth = 11in}}
    { \geometry{top=.5in,left=.5in,right=.5in,bottom=.5in} }
	{\ifthenelse{ \lengthtest{ \paperwidth = 297mm}}
		{\geometry{top=1cm,left=1cm,right=1cm,bottom=1cm} }
		{\geometry{top=1cm,left=1cm,right=1cm,bottom=1cm} }
	}
\pagestyle{plain}
\makeatletter
\renewcommand{\section}{\@startsection{section}{1}{0mm}%
                                {-1ex plus -.5ex minus -.2ex}%
                                {0.5ex plus .2ex}%x
                                {\normalfont\large\bfseries}}
\renewcommand{\subsection}{\@startsection{subsection}{2}{0mm}%
                                {-1explus -.5ex minus -.2ex}%
                                {0.5ex plus .2ex}%
                                {\normalfont\normalsize\bfseries}}
\renewcommand{\subsubsection}{\@startsection{subsubsection}{3}{0mm}%
                                {-1ex plus -.5ex minus -.2ex}%
                                {1ex plus .2ex}%
                                {\normalfont\small\bfseries}}
\makeatother
\setcounter{secnumdepth}{0}
\setlength{\parindent}{0pt}
\setlength{\parskip}{0pt plus 0.5ex}
% -----------------------------------------------------------------------

\title{ST2334 Cheatsheet}

\begin{document}

\raggedright
\footnotesize

\begin{center}
     \Large{\textbf{ST2334 Midterm Cheatsheet}} \\
\end{center}
\begin{multicols}{3}
\setlength{\premulticols}{1pt}
\setlength{\postmulticols}{1pt}
\setlength{\multicolsep}{1pt}
\setlength{\columnsep}{2pt}

\section{Basic probability concepts}
\subsection{Observation}
Any recording of information, whether it is numerical or categorical.

\subsection{Statistical Experiment}
Any procedure that generates a set of data (observations).

\subsection{Sample Space}
The set of all possible outcomes of a statistical experiment is called the \textbf{sample space} and it is represented by the symbol $S$.

\subsection{Sample Point}
Every outcome in a sample space is called an element of the sample space or simply a sample point.

\subsection{Event}
An event is a subset of a sample space.

\subsection{Simple Event}
An event is said to be simple if it consists of exactly one outcome (i.e. one sample point)

\subsection{Compound Event}
An event is said to be compound if it consists of more than one outcomes (or sample points).

\subsection{}
1. The sample space is itself an event and is usually called a sure event. \\
2. A subset of $S$ that contains no elements at all is the empty set, denoted by $\emptyset$ , and is usually called a null event.


\section{Operations of Events}

\subsection{Union}
The union of two events A and B, denoted by $A \cup B$, is the event containing all the elements that belong to A or B or to both. That is,
$$
A \cup B=\{x : x \in A \text { or } x \in B\}
$$

\subsection{Intersection}
The intersection of two events A and B, denoted by $A \cap B$ or simply $AB$, is the event containing all elements that are common to A and B. That is
$$
A \cap B=\{x : x \in A \text { and } x \in B\}
$$

\subsection{Complement}
The complement of event A with respect to S, denoted by $A'$ or $A^{C}$, is the set of all elements of S that are not in A. That is
$$
A^{\prime}=\{x : x \in S \text { and } x \notin A\}
$$

\subsection{Mutually Exclusive Events}
Two events A and B are said to be mutually exclusive or mutually disjoint if $A \cap B = \emptyset$, that is, if A and B have no elements in common.

\subsection{Union of $n$  Events}
The union of n events $A_1 ,A_2 , \cdots ,A_n$ , denoted by
$$
A_1 \cup A_2 \cup ... \cup A_n
$$
is the event containing all the elements that belong to one or more of the events $A_1$, $A_2$, or ..., or $A_n$. That is
$$
\bigcup_{i=1}^{n} A_{i}=A_{1} \cup A_{2} \cup \cdots \cup A_{n}=\left\{x : x \in A_{1} \text { or } \cdots \text { or } x \in A_{n}\right\}
$$

\subsection{Intersection of $n $ Events}
The intersection of n events $A_1 ,A_2 , \cdots ,A_n$ , denoted by
$$
A_1 \cap A_2 \cap ... \cap A_n
$$
is the event containing all the elements that are common to all of the events $A_1$, $A_2$, or ..., or $A_n$. That is
$$
\bigcap_{i=1}^{n} A_{i}=A_{1} \cap A_{2} \cap \cdots \cap A_{n}=\left\{x : x \in A_{1} \text { and } \cdots \text { and } x \in A_{n}\right\}
$$

\section{Counting}

\subsection{Permutation}
A permutation is an arrangement of r objects from a set of n objects, where $r \leq n$.
(Note that the \textbf{order is taken into consideration} in permutation.)
$$
_{n} P_{r}=n(n-1)(n-2) \cdots(n-(r-1))=n ! /(n-r) !
$$
When not all objects are distinct,
$$
n P_{n_{1}, n_{2}, \cdots, n_{k}}=\frac{n !}{n_{1} ! n_{2} ! \cdots n_{k} !}
$$
When in circle: $(n - 1)!$


\subsection{Combination}
the number of ways of selecting r objects from n objects \textbf{without regard to the order}.
$$
\left(\begin{array}{l}{n} \\ {r}\end{array}\right)=\frac{n !}{r !(n-r) !}
$$

\section{Axioms of Probability}

\subsection{Axiom 1}
$0 \leq Pr(A) \leq 1$

\subsection{Axiom 2}
$Pr(S) = 1$

\subsection{Axiom 3}
If $A_1, A_2, \cdots$ are mutually exclusive (disjoint) events, then

$$
\operatorname{Pr}\left(\bigcup_{i=1}^{\infty} A_{i}\right)=\sum_{i=1}^{\infty} \operatorname{Pr}\left(A_{i}\right)
$$

\subsection{Inclusion-Exclusion Principle}
$$
\begin{array}{l}{\operatorname{Pr}\left(A_{1} \cup A_{2} \cup \cdots \cup A_{n}\right)=\sum_{i=1}^{n} \operatorname{Pr}\left(A_{i}\right)-\sum_{i=1}^{n-1} \sum_{j=i+1}^{n} \operatorname{Pr}\left(A_{i} \cap A_{j}\right)} \\ {\quad+\sum_{i=1}^{n-2} \sum_{j=i+1}^{n-1} \sum_{k=j+1}^{n} \operatorname{Pr}\left(A_{i} \cap A_{j} \cap A_{k}\right)-\cdots \cdots} \\ {\quad+(-1)^{n+1} \operatorname{Pr}\left(A_{1} \cap A_{2} \cap \cdots \cap A_{n}\right)}\end{array}
$$

\subsection{Conditional Probability}
The conditional probability of B given A is defined as
$$
\operatorname{Pr}(B | A)=\frac{\operatorname{Pr}(A \cap B)}{\operatorname{Pr}(A)}, \quad \text { if } \operatorname{Pr}(A) \neq 0
$$

\subsection{Multiplication Rule of Probability}
In general,
$$
\begin{array}{c} 
\\{\operatorname{Pr}\left(A_{1} \cap \cdots \cap A_{n}\right)=\operatorname{Pr}\left(A_{1}\right) \operatorname{Pr}\left(A_{2} | A_{1}\right)} 
\\ {\quad \quad \operatorname{Pr}\left(A_{3} | A_{1} \cap A_{2}\right) \cdots \operatorname{Pr}\left(A_{n} | A_{1} \cap \cdots \cap A_{n-1}\right)} 
\\ {\text { provided that } \operatorname{Pr}\left(A_{1} \cap \cdots \cap A_{n-1}\right)>0}\end{array}
$$

\subsection{The Law of Total Probability}

Let $A_1, A_2, \cdots , A_n$ be  a partition of the sample space $S$. That is  $A_1, A_2, \cdots , A_n$ are mutually exclusive and exhaustive events such that $A_i \cap A_j = \emptyset$ for $i \neq j$ and $\bigcup_{i=1}^{n} A_{i}=S$.\\
Then for any event B
$$
\operatorname{Pr}(B)=\sum_{i=1}^{n} \operatorname{Pr}\left(B \cap A_{i}\right)=\sum_{i=1}^{n} \operatorname{Pr}\left(A_{i}\right) \operatorname{Pr}\left(B | A_{i}\right)
$$

\subsection{Bayes' Theorem}
Let $A_1, A_2, \cdots , A_n$ be  a partition of the sample space $S$. Then
$$
\operatorname{Pr}\left(A_{k} | B\right)=\frac{\operatorname{Pr}\left(A_{k}\right) \operatorname{Pr}\left(B | A_{k}\right)}{\sum_{i=1}^{n} \operatorname{Pr}\left(A_{i}\right) \operatorname{Pr}\left(B | A_{i}\right)}
$$
for $k = 1, \cdots, n$. Or
$$
\operatorname{Pr}\left(A_{k} | B\right)=\frac{\operatorname{Pr}\left(A_{k}\right) \operatorname{Pr}\left(B | A_{k}\right)}{\operatorname{Pr}(B)}
$$

\subsection{Independent Events}
Two events A and B are independent iff
$$
\operatorname{Pr}(A \cap B)=\operatorname{Pr}(A) \operatorname{Pr}(B)
$$

\subsection{Pairwise Independence}
A set of events $A_1, A_2, \cdots , A_n$ are pairwise independent iff
$$
\operatorname{Pr}(A_i \cap A_j)=\operatorname{Pr}(A_i) \operatorname{Pr}(A_j)
$$
for $i \neq j$ and $i, j = 1, \cdots, n$

\subsection{Mutual Independence}
A set of events $A_1, A_2, \cdots , A_n$ are mutually independent iff for any subset $\{A_{i1}, A_{i2}, \cdots , A_{ik}\}$ of $A_1, A_2, \cdots , A_n$,

$$
\operatorname{Pr}\left(A_{i_{1}} \cap A_{i_{2}} \cap \cdots \cap A_{i_{k}}\right)=\operatorname{Pr}\left(A_{i_{1}}\right) \operatorname{Pr}\left(A_{i_{2}}\right) \cdots \operatorname{Pr}\left(A_{i_{k}}\right)
$$

Note: their complements are also mutually independent.

\section{Concepts of Random Variables}

\subsection{Random Variable}
Let S be a sample space associated with the experiment, E. A function X, which assigns a number to every element $s \in S$, is called a random variable.

\subsection{Discrete Random Variable}
If the number of possible values of $X$(i.e.,$R_X$,the range space) is \textbf{finite or countably infinite}, we call $X$ a discrete random variable.

\subsection{Probability (Mass) Function}
The probability of $X = x_i$ denoted by $f(x_i)$ (i.e. $f(x_i) = \operatorname{Pr} \left(X = x_i\right)$, must satisfy the following two conditions. \\
(1) $f(x_i) \geq 0$ for all $x_i$. \\
(2)$\sum_{i=1}^{\infty} f\left(x_{i}\right)=1$

\subsection{Continuous Random Variable}
The range space $R_x$ is an interval or a range of intervals.

\subsection{Probability Density Function}
Let X be a \textbf{continuous} random variable.
$$
\begin{array}{l}{\text { 1. } f(x) \geq 0 \text { for all } x \in R_{X}} 
\\ {\text { 2. } \int_{R_{X}} f(x) d x=1 \text { or } \int_{-\infty}^{\infty} f(x) d x=1} 
\\ {\quad \text { since } f(x)=0 \text { for } x \text { not in } R_{X}}
\\{\left.\text { 3. For any } c \text { and } d \text { such that } c<d, \text { (i.e. }(c, d) \subset \boldsymbol{R}_{X}\right) \text { , }} \\ {\qquad \operatorname{Pr}(c \leq X \leq d)=\int_{c}^{d} f(x) d x}\end{array}
$$

\subsection{Cumulative Distribution Function}
We define $F(x)$ to be the \textbf{cumulative distribution function} of the random variable $X$ (abbreviated as c.d.f.)
where
$$
F(x) = \operatorname{Pr}\left(X \leq x \right)
$$

If $X$ is a \textbf{discrete} random variable, then its c.d.f is a step function.
$$
\begin{aligned} F(x) &=\sum_{t \leq x} f(t) \\ &=\sum_{t \leq x} \operatorname{Pr}(X=t) \end{aligned}
$$

If $X$ is a \textbf{continuous} random variable, then
$$
F(x)=\int_{-\infty}^{x} f(t) d t
$$
For a \textbf{continuous} random variable $X$,
$$
f(x)=\frac{d F(x)}{d x}
$$
if the derivative exists.

\subsection{Mean}
If $X$ is a \textbf{discrete} random variable, taking on values $x_1, x_2, \cdots$ with probability function $f(x)$, then the mean or expected value of $X$, denoted by $E(X)$, is defined by
$$
\mu_{X}=E(X)=\sum_{i} x_{i} f\left(x_{i}\right)=\sum_{x} x f(x)
$$
If $X$ is a \textbf{continuous} random variable with probability density function $f(x)$, then the mean is defined by
$$
\mu_{X}=E(X)=\int_{-\infty}^{\infty} x f(x) d x
$$
For any function $g(X)$, \\
(a) $E[g(X)]=\sum_{x} g(x) f_{X}(x)$ \\
(b) $E[g(X)]=\int_{-\infty}^{\infty} g(x) f_{X}(x) d x$ \\
Property:
$$
E(aX + b) = a E(X) + b
$$
In general,
$$
\begin{array}{l}{E\left[a_{1} g_{1}(X)+a_{2} g_{2}(X)+\cdots+a_{k} g_{k}(X)\right]} \\ {\quad=a_{1} E\left[g_{1}(X)\right]+a_{2} E\left[g_{2}(X)\right]+\cdots+a_{k} E\left[g_{k}(X)\right]}\end{array}
$$

\subsection{Variance}
$$
\begin{aligned} & \sigma_{X}^{2}=V(X)=E\left[\left(X-\mu_{X}\right)^{2}\right] \\=&\left\{\begin{array}{ll}{\sum_{x}\left(x-\mu_{X}\right)^{2} f_{X}(x),} & {\text { if } X \text { is discrete }} \\ {\int_{-\infty}^{\infty}\left(x-\mu_{X}\right)^{2} f_{X}(x) d x,} & {\text { if } X \text { is continuous. }}\end{array}\right.\end{aligned}
$$
Remarks:
$$
\begin{array}{l}{\text { (a) } V(X) \geq 0} \\ {\text { (b) } V(X)=E\left(X^{2}\right)-[E(X)]^{2}}\end{array}
$$
Property:
$$
V(a X+b)=a^{2} V(X)
$$

\subsection{Standard Deviation}
The \textbf{positive square root} of the variance.

\subsection{Moment}
The \textbf{k-th moment} of X is defined by $E(X^k)$.

\subsection{Chebyshev’s Inequality}
Let $X$ be a random variable (discrete or continuous) with $E(X) = \mu$ and $V(X) = \sigma ^2$.
For any positive number $k$,
$$
\operatorname{Pr}(|X-\mu| \geq k \sigma) \leq 1 / k^{2}
$$
That is, the probability that the value of X lies at least $k$ standard deviation from its mean is at most $\frac{1}{k^2}$. \\
Alternatively,
$$
\operatorname{Pr}(|X-\mu|<k \sigma) \geq 1-1 / k^{2}
$$
This is true for \textbf{all} distributions with finite mean and variance. 

\section{Two-dimensional Random Variables}

\subsection{Definition of 2D RV}
Let E be an experiment and S a sample space associated with E. Let X and Y be two functions each assigning a real number to each $s \in S$. \\
We call $(X, Y)$ a \textbf{two-dimensional random variable}. (Sometimes called a \textbf{random vector}). \\
The above definition can be extended to $n$ random variables.

\subsection{Range Space}
$$R_{X,Y} = \{(x, y) | x = X(s), y = Y(s), s \in S\}$$
The above definition can be extended to more than two random variables.

\subsection{Discrete vs. Continuous}
\textbf{Discrete}: $(X, Y)$ is a two-dimensional \textbf{discrete} random variable if the possible values of $(X(s), Y(s))$ are \textbf{finite or countably infinite}. \\
\textbf{Continuous}: 
$(X, Y)$ is a two-dimensional \textbf{continuous} random variable if the possible values of $(X(s), Y(s))$ can assume all values in some region of the Euclidean plane $\mathbb{R}^2$.

\subsection{Joint Probability Function}
Let $(X, Y)$ be a 2-dimensional \textbf{discrete} random variable defined on the sample space of an experiment. With each possible value $(x_i, y_i)$, we associate a number $f_{X,Y}(x_i, y_i)$ representing $\operatorname{Pr}(X = x_i, Y = y_i)$ and satisfying the following conditions: \\
1. $f_{X,Y}(x_i, y_j) \geq 0$ for all $(x_i, y_j) \in R_{X,Y} $. \\
2.
$$
\sum_{i=1}^{\infty} \sum_{j=1}^{\infty} f_{X, Y}\left(x_{i}, y_{j}\right)=\sum_{i=1}^{\infty} \sum_{j=1}^{\infty} \operatorname{Pr}\left(X=x_{i}, Y=y_{j}\right)=1
$$

The function $f_{X,Y}(x, y)$ is called the \textbf{joint probability function} for $(X, Y)$.

$$
\begin{array}{l}
\operatorname{Pr}((X, Y) \in A)=\underbrace{\sum \sum} f_{X, Y}(x, y) 
\\ \hspace*{\fill} (x,y) \in A \hspace{1.2cm}
\end{array}
$$

\subsection{Joint Probability Density Function}
Let $(X, Y)$ be a 2-dimensional \textbf{continuous} random variable assuming all values in some region $R$ of the Euclidean plane $\mathbb{R}^2$. $f_{X,Y}(x, y)$ is called a \textbf{joint probability density function} if it satisfies the following conditions:
$$
\begin{array}{l}{\text { 1. } f_{X, Y}(x, y) \geq 0 \text { for all }(x, y) \in R_{X, Y}} \\ {\text { 2. }} \\ {\qquad \iint_{(x, y) \in R_{X, Y}} f_{X, Y}(x, y) d x d y=1} \\ {\text { or }} \\ {\qquad \int_{-\infty}^{\infty} \int_{-\infty}^{\infty} f_{X, Y}(x, y) d x d y=1}\end{array}
$$


\vfill
\hrule
~\\
Prepared by Zechu, AY2019/2020 Semester 1
\end{multicols}

\end{document}